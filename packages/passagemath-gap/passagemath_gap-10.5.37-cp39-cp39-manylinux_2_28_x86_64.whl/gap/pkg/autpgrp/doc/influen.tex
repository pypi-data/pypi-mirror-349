%%%%%%%%%%%%%%%%%%%%%%%%%%%%%%%%%%%%%%%%%%%%%%%%%%%%%%%%%%%%%%%%%%%%%%%%%
%%
%A  influen.tex        AutPGrp documentation                 Bettina Eick
%A                                                         Eamonn O'Brien
%%

%%%%%%%%%%%%%%%%%%%%%%%%%%%%%%%%%%%%%%%%%%%%%%%%%%%%%%%%%%%%%%%%%%%%%%%%%
\Chapter{Influencing the algorithm}

A number of choices can be made by the user to influence the performance 
of `AutomorphismGroupPGroup'. Below we identify these choices 
and their default values used in `AutomorphismGroup'. We use the optional
argument <flag> of `AutomorphismGroupPGroup' to invoke user-defined choices.
The possible values for <flag> are

\beginitems
`<flag> = false' & the user-defined defaults are employed in the algorithm.
                 See below for a list of possibilities.

`<flag> = true'  & invokes the interactive version of the algorithm
                 as described in Section "An interactive version of
                 the algorithm".
\enditems

In the next section we give a brief outline of the algorithm which is 
necessary to understand the possible choices. Then we introduce the 
choices the later sections of this chapter.

%%%%%%%%%%%%%%%%%%%%%%%%%%%%%%%%%%%%%%%%%%%%%%%%%%%%%%%%%%%%%%%%%%%%%
\Section{Outline of the algorithm}

The basic algorithm proceeds by induction 
down the lower $p$-central series of a given $p$-group $G$; that is, it 
successively computes $Aut(G_i)$ for the quotients $G_i = G / P_i(G)$ of 
the descending sequence of subgroups defined by $P_1(G) = G$ and 
$P_{i+1}(G)=[P_i(G),G] P_i(G)^p$ for $i\geq 1$. Hence, in the initial 
step of the algorithm, $Aut(G_2) = GL(d,p)$ where $d$ is the rank of 
the elementary abelian group $G_2$. In the inductive step it determines 
$Aut(G_{i+1})$ from $Aut(G_i)$. For this purpose we introduce
an action of $Aut(G_i)$ on a certain elementary abelian $p$-group $M$ 
(the *$p$-multiplicator* of $G_i$). The main computation of the inductive 
step is the determination of the stabiliser in $Aut(G_i)$ of a subgroup 
$U$ of $M$ (the *allowable subgroup* for $G_{i+1}$). This stabiliser 
calculation is the bottle-neck of the algorithm.

Our package incorporates a number of refinements designed to simplify
this stabiliser computation. Some of these refinements incur overheads
and hence they are not always invoked. The features outlined below 
allow the user to select them. 

Observe that the initial step of the algorithm returns $GL(d,p)$. But 
$Aut(G)$ may induce on $G_2$ a proper subgroup, say $K$, of $GL(d,p)$.
Any intermediate subgroup of $GL(d,p)$ which contains $K$ suffices for
the algorithm and we supply two methods to construct a suitable subgroup:
these use characteristic subgroups or invariants of normal subgroups of $G$. 
(See Section "The initialisation step".)

In the inductive step an action of $Aut(G_i)$ on an elementary abelian 
group $M$ is used. This action is computed as a matrix action on a vector 
space. To simplify the orbit-stabiliser computation of the subspace $U$ 
of $M$, we can construct the stabiliser of $U$ by iteration over a sequence 
of $Aut(G_i)$-invariant subspaces of $M$. 
(See Section "Stabilisers in matrix groups".)

Orbit-stabiliser computations in finite solvable groups given by a
polycyclic generating sequence are much more efficient than generic
computations of this type. Thus our algorithm makes use of a large 
solvable normal subgroup $S$ of $Aut(G_i)$. Further, it is useful if
the generating set of $Aut(G_i)$ outside $S$ is as small as possible. 
To achieve this we determine a permutation representation of $Aut(G_i)/S$ 
and use this to reduce the number of generators if possible. (See Section 
"Searching for a small generating set".)

%%%%%%%%%%%%%%%%%%%%%%%%%%%%%%%%%%%%%%%%%%%%%%%%%%%%%%%%%%%%%%%%%%%%%
\Section{The initialisation step} 

Assume  we seek to  compute the  automorphism group of a  $p$-group $G$
having Frattini rank  $d$.  We first determine as small as possible a
subgroup of $GL(d, p)$ whose extension can act on $G$. 

The user can choose the initialisation routine by assigning 
`InitAutGroup' to any one of the following.

\beginitems
`InitAutomorphismGroupOver' & to use the minimal overgroups;

`InitAutomorphismGroupChar' & to use the characteristic subgroups;

`InitAutomorphismGroupFull' & to use the full $GL(d,p)$.
\enditems

*a) Minimal Overgroups*

We  determine the  minimal over-groups  of the Frattini subgroup of 
$G$ and compute invariants of these which must be respected by the 
automorphism group  of $G$. We partition the minimal overgroups and 
compute the stabiliser in $GL(d, p)$ of this partition.

The partition of the minimal overgroups is computed using the
function `PGFingerprint( G, U )'. This is the time-consuming
part of this initialisation method. The user can 
overwrite the function `PGFingerprint'.

*b) Characteristic  Subgroups*

Compute  a  generating   set  for  the stabiliser in  $GL (d, p)$ of  
a chain of  characteristic subgroups of $G$. In  practice, we construct 
a characteristic  chain by determining 2-step  centralisers  and omega  
subgroups  of  factors  of the  lower $p$-central series.  

However, there are often other characteristic subgroups which are not
found by these approaches. The user can overwrite the function
`PGCharSubgroups( G )' to supply a set of characteristic subgroups.

*c) Defaults*

In the method for `AutomorphismGroup' we use a default strategy:
if the value $\frac{p^d-1}{p-1}$ is less than 1000, then we 
use the minimal overgroup approach, otherwise the characteristic 
subgroups are employed. An exception is made for homogeneous abelian
groups where we initialise the algorithm with the full group $GL(d,p)$.

%%%%%%%%%%%%%%%%%%%%%%%%%%%%%%%%%%%%%%%%%%%%%%%%%%%%%%%%%%%%%%%%%%%%%
\Section{Stabilisers in matrix groups}

Consider the $i$th inductive step of the algorithm. Here $A \leq
Aut(G_i)$ acts as matrix group on the elementary abelian $p$-group
$M$ and we want to determine the stabiliser of a subgroup $U \leq M$.

We use the MeatAxe to compute a series of $A$-invariant subspaces
through $M$ such that each factor in the series is irreducible as
$A$-module. Then we use this series to break the computation 
of $Stab_A(U)$ into several smaller orbit-stabiliser calculations.

Note that a theoretic argument yields an $A$-invariant subspace
of $M$ a priori: the nucleus $N$. This is always used to split
the computation up. However, it may happen that $N = M$ and hence
results in no improvement. 

\>`CHOP_MULT' V

The invariant series through $M$ is computed and used if the 
global variable `CHOP_MULT' is set to `true'. Otherwise, the algorithm
tries to determine $Stab_A(U)$ in one step. By default, `CHOP_MULT'
is `true'.

%%%%%%%%%%%%%%%%%%%%%%%%%%%%%%%%%%%%%%%%%%%%%%%%%%%%%%%%%%%%%%%%%%%%%
\Section{Searching for a small generating set}

After each step of the computation, we attempt to  find a nice generating
set for the automorphism group of the current factor.

If the automorphism group is soluble, we store a polycyclic generating
set; otherwise,  we store  such a generating  set for a  large soluble
normal  subgroup $S$ of the automorphism group $A$, and as few generators 
outside  as  possible. If $S = A$ and a polycyclic  generating set for
$S$ is known,  many steps  of  the algorithm proceed more rapidly. 

\>`NICE_STAB' V

It may be both time-consuming and difficult to reduce the number of 
generators for $A$ outside $S$. Note that if the initialisation of the 
algorithm is by `InitAutomorphismGroupOver', then we always know a 
permutation representation for $A/S$. Occasionally the search for 
a small generating set is expensive.  If this is  observed, one 
could set the flag `NICE_STAB' to  `false'  and  the  algorithm  no  
longer  invokes this search.

%%%%%%%%%%%%%%%%%%%%%%%%%%%%%%%%%%%%%%%%%%%%%%%%%%%%%%%%%%%%%%%%%%%%%
\Section{An interactive version of the algorithm}

The choice of initialisation and the choice of chopping of the 
$p$-multiplicator can also be driven by an interactive version
of the algorithm. We give an example.

\beginexample
gap> G := SmallGroup( 2^8, 1000 );;
gap> SetInfoLevel( InfoAutGrp, 3 );

gap> AutomorphismGroupPGroup( G, true );
#I  step 1: 2^3 -- init automorphisms 

choose initialisation (Over/Char/Full):     # we choose Full 
#I    init automorphism group : Full 
#I  step 2: 2^3 -- aut grp has size 168
#I    computing cover
#I    computing matrix action
#I    computing stabilizer of U
#I    dim U = 3  dim N = 6  dim M = 6

chop M/N and N: (y/n):                      # we choose n
#I    induce autos and add central autos
#I  step 3: 2^2 -- aut grp has size 12288
#I    computing cover
#I    computing matrix action
#I    computing stabilizer of U
#I    dim U = 6  dim N = 5  dim M = 8

chop M/N and N: (y/n):                      # we choose y 
#I    induce autos and add central autos
#I  final step: convert
rec( 
  glAutos := [ Pcgs([ f1, f2, f3, f4, f5, f6, f7, f8 ]) -> [ f1, f2*f3, f3, 
          f4, f5, f6*f7, f7, f8 ], 
      Pcgs([ f1, f2, f3, f4, f5, f6, f7, f8 ]) -> 
        [ f1*f3*f5*f6, f2*f3, f3, f4, f5*f8, f6*f7, f7, f8 ], 
      Pcgs([ f1, f2, f3, f4, f5, f6, f7, f8 ]) -> 
        [ f1*f3, f2*f4, f3, f4*f7, f5*f7, f6*f7*f8, f7, f8 ] ], glOrder := 4, 
  agAutos := 
    [ Pcgs([ f1, f2, f3, f4, f5, f6, f7, f8 ]) -> [ f1*f4, f2, f3, f4*f8, f5, 
          f6, f7, f8 ], Pcgs([ f1, f2, f3, f4, f5, f6, f7, f8 ]) -> 
        [ f1, f2*f4, f3, f4*f7, f5, f6*f7*f8, f7, f8 ], 
      Pcgs([ f1, f2, f3, f4, f5, f6, f7, f8 ]) -> 
        [ f1*f5, f2, f3, f4, f5, f6, f7, f8 ], 
      Pcgs([ f1, f2, f3, f4, f5, f6, f7, f8 ]) -> 
        [ f1, f2*f5, f3, f4, f5, f6, f7, f8 ], 
      Pcgs([ f1, f2, f3, f4, f5, f6, f7, f8 ]) -> 
        [ f1, f2, f3*f5, f4, f5, f6, f7, f8 ], 
      Pcgs([ f1, f2, f3, f4, f5, f6, f7, f8 ]) -> 
        [ f1*f6, f2, f3, f4, f5*f7*f8, f6, f7, f8 ], 
      Pcgs([ f1, f2, f3, f4, f5, f6, f7, f8 ]) -> 
        [ f1, f2*f6, f3, f4*f7*f8, f5, f6, f7, f8 ], 
      Pcgs([ f1, f2, f3, f4, f5, f6, f7, f8 ]) -> 
        [ f1*f8, f2, f3, f4, f5, f6, f7, f8 ], 
      Pcgs([ f1, f2, f3, f4, f5, f6, f7, f8 ]) -> 
        [ f1, f2*f8, f3, f4, f5, f6, f7, f8 ], 
      Pcgs([ f1, f2, f3, f4, f5, f6, f7, f8 ]) -> 
        [ f1, f2, f3*f8, f4, f5, f6, f7, f8 ], 
      Pcgs([ f1, f2, f3, f4, f5, f6, f7, f8 ]) -> 
        [ f1*f7, f2, f3, f4, f5, f6, f7, f8 ], 
      Pcgs([ f1, f2, f3, f4, f5, f6, f7, f8 ]) -> 
        [ f1, f2*f7, f3, f4, f5, f6, f7, f8 ], 
      Pcgs([ f1, f2, f3, f4, f5, f6, f7, f8 ]) -> 
        [ f1, f2, f3*f7, f4, f5, f6, f7, f8 ] ], 
  agOrder := [ 2, 2, 2, 2, 2, 2, 2, 2, 2, 2, 2, 2, 2 ], 
  one := IdentityMapping( <pc group of size 256 with 8 generators> ), 
  group := <pc group of size 256 with 8 generators>, size := 32768 )
\endexample

Two points are worthy of comment.
First, the interactive version of the algorithm permits the user to 
make a suitable choice in each step of the algorithm instead of making
one choice at the beginning. Secondly, the output of the `Info' function 
shows the ranks of the $p$-multiplicator and allowable subgroup,
and thus allow the user to observe the scale of difficulty 
of the computation. 

%%%%%%%%%%%%%%%%%%%%%%%%%%%%%%%%%%%%%%%%%%%%%%%%%%%%%%%%%%%%%%%%%%%%%
\Section{Acknowledgements}

We thank Alexander Hulpke for helping us with efficiency 
problems. Werner Nickel provided some functions from 
the {\GAP} `PQuotient' which are used in this package.

%%%%%%%%%%%%%%%%%%%%%%%%%%%%%%%%%%%%%%%%%%%%%%%%%%%%%%%%%%%%%%%%%%%%%
